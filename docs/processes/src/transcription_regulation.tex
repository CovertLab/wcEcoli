\documentclass[12pt]{article}
\usepackage[ruled,vlined,noresetcount]{algorithm2e}
\usepackage{amsmath}

\topmargin 0.0cm
\oddsidemargin 0.2cm
\textwidth 16cm
\textheight 21cm
\footskip 1.0cm

\begin{document}

\baselineskip24pt

\paragraph{Transcription regulation}
\label{sec:transcription_reg}

\subparagraph{Model implementation.}
There are two aspects to modeling transcriptional regulation: (1) modeling the activation or inhibition of a transcription factor (e.g., by a ligand), and (2) given an active transcription factor, modeling its effect on RNA polymerase recruitment to a promoter site.  The the enhanced coverage of the regulatory network - 438 regulatory interactions described by 22 transcription factors that regulate 355 genes - is a significant difference from the \emph{M. genitalium} model. To incorporate this network, regulation is represented by three different classes of transcription regulators: zero-component systems, one-component systems and two-component systems.

\subparagraph{Modeling transcription factor activation.}
We consider three classes of transcription factors based on their mechanism of activation:

\begin{enumerate}
\item \textbf{Zero-component systems}: transcription factors that are considered to be active whenever they are expressed.  Examples include the Fis and Hns proteins.  These two proteins, for instance, are important in maintaining higher-order DNA structure and likely have complex feedback loops modulating their activity.  Because this complexity is not yet fully understood, we make the simplifying assumption that these proteins are always active unless they are knocked out. Zero-component systems are modeled in the \texttt{TfBinding} process in the model, which handles transcription factor binding to promoters.

\item \textbf{One-component systems}: transcription factors that are directly activated or inhibited by a small molecule ligand.  Examples of this class include the repressor TrpR which binds tryptophan, and the inducer AraC which binds arabinose. One-component systems are modeled in the \texttt{TfBinding} and \texttt{Equilibrium} processes in the model, which handle transcription factor binding to promoters and transcription factor binding to ligands, respectively.

\item \textbf{Two-component systems}: transcription factors that are paired with a separate sensing protein that responds to an environmental stimulus (these are simple analogs to the vast, complicated signaling networks that exist in eukaryotic cells).  The sensing protein phosphorylates the cognate transcription factor in a condition-dependent fashion.  Examples include ArcA which is phosphorylated by its cognate ArcB in anaerobic conditions, and NarL which responds to the presence of nitrate when phosphorylated by its cognate sensor NarX. Two-component systems are modeled in the \texttt{TfBinding}, \texttt{Equilibrium}, and \texttt{TwoComponentSystems} processes in the model, which handle transcription factor binding to promoters, transcription factor binding to ligands, and phosphotransfer reactions of signaling pathways, respectively.
\end{enumerate}

\subparagraph{Zero-component systems.}
We assume all transcription factors of this class will bind to available promoter sites.

\subparagraph{One-component systems.}
For a transcription factor with concentration \(T\) whose activity is directly modulated by a ligand with concentration \(L\) that binds with stoichiometry \(n\), we assume that the two species achieve equilibrium on a short time scale and that the affinity of the two molecules can be described by a dissociation constant \(K_d\):

\begin{equation}
nL + T \rightleftharpoons T^*
\end{equation}

\noindent where \(T^*\) represents the concentration of the ligand-bound transcription factor.

With the dissociation constant \(K_d\) defined as:

\begin{equation}
K_d = \frac{L^n \cdot T}{T^*}
\end{equation}

\noindent we have:

\begin{equation}
\begin{aligned}
\frac{T^*}{T_T} &= \frac{L^n}{L^n + K_d} \\
\end{aligned}
\label{eq:transcription_1cs}
\end{equation}

\noindent where $T_T$ is the total concentration of the transcription factor, both ligand-bound and unbound. As we can see, the fraction of bound transcription factor is a function of ligand concentration and the dissociation constant.  Importantly, if the ligand concentration is (approximately) constant over time, the fraction of bound transcription factor is (approximately) constant over time.

To computationally simulate this model we start with total counts of free transcription factor and ligand, completely dissociated from one another.  We then form one molecule of the ligand-TF complex at a time and evaluate how close the ratio of \(L^n \cdot T / T^*\) is to the actual \(K_d\).  We select the values of \(L\), \(T\) and \(T^*\) that minimize the absolute difference between \(K_d\) and \(L^n \cdot T / T^*\) (see Algorithm~\ref{equilibrium_binding_algorithm}).

\subparagraph{Two-component systems.}
For a transcription factor with concentration \(T\); a cognate sensing protein with concentration \(S\); a ligand with concentration \(L\); subscripts \(f\) denoting a free (unbound) form of a molecule, \(b\) denoting a ligand-bound form of a molecule, and \(p\) denoting a phosphorylated form of a molecule; and \(ATP\), \(ADP\), \(H^+\), and \(H_2O\) denoting concentrations of these molecules, we propose a system with the following:

Free (unbound) cognate sensing protein at equilibrium with ligand-bound cognate sensing protein, described by dissociation constant \(K_d\):

\begin{equation}
L + S_{f} \rightleftharpoons S_{b}
\end{equation}

\noindent The autophosphorylation of a free (unbound) cognate sensing protein at a rate \(k_A\):

\begin{equation}
S_{f} + ATP \overset{k_A}{\rightarrow} S_{fp} + ADP + H^+
\end{equation}

\noindent The autophosphorylation of a ligand-bound cognate sensing protein at a rate \(k_B\):

\begin{equation}
S_{b} + ATP \overset{k_B}{\rightarrow} S_{bp} + ADP + H^+
\end{equation}

\noindent The phosphorylation of a transcription factor by its free, phosphorylated cognate sensing protein at a rate \(k_C\):

\begin{equation}
S_{fp} + T \overset{k_C}{\rightarrow} S_{f} + T_{p}
\end{equation}

\noindent The phosphorylation of a transcription factor by its bound, phosphorylated cognate sensing protein at a rate \(k_D\):

\begin{equation}
S_{bp} + T \overset{k_D}{\rightarrow} S_{b} + T_{p}
\end{equation}

\noindent The auto-phosphatase activity of a transcription factor at a rate \(k_E\):

\begin{equation}
T_{p} + H_2O \overset{k_E}{\rightarrow} T + P_i
\end{equation}

\noindent Ligand binding is simulated in a fashion identical to the one-component systems. By assuming mass-action kinetics, we can represent the rest of this system mathematically using ordinary differential equations:
\begin{align}
    \frac{dS_f}{dt} &= -k_A \cdot S_f \cdot ATP + k_C \cdot S_{fp} \cdot T \\
    \frac{dS_b}{dt} &= -k_B \cdot S_b \cdot ATP + k_D \cdot S_{bp} \cdot T \\
    \frac{dT}{dt} &= -k_C \cdot S_{fp} \cdot T - k_D \cdot S_{bp} \cdot T + k_E \cdot T_p \cdot H_2O
\end{align}
\begin{align}
    \frac{dS_{fp}}{dt} &= -\frac{dS_f}{dt} \\
    \frac{dS_{bp}}{dt} &= -\frac{dS_b}{dt} \\
    \frac{dT_p}{dt} &= -\frac{dT}{dt}
\end{align}
\noindent This system of equations is simulated using a numerical ODE integrator (see Algorithm~\ref{two_component_systems_algorithm}).

\subparagraph{Modeling the modulation of RNA polymerase recruitment.}
After modeling transcription factor activation, we need to model the probability that the transcription factor is bound to DNA, \(p_T\), and, when the transcription factor is DNA-bound, its effect on RNA polymerase recruitment to the promoter site, \(\Delta r\) (see Algorithm~\ref{tf_binding_algorithm}).  Recalling the notation used in the \textit{Transcription} section (Algorithm~\ref{transcript_initiation_algorithm}), we want to modulate the \(j^{th}\) entry in the \(v_\text{synth}\) vector of RNA polymerase initiation probabilities such that:

\begin{equation}
v_{\text{synth}, j} = \alpha_j + \sum_{i} p_{T, i} \Delta r_{i j}
\end{equation}

\noindent where \(\alpha_j\) represents basal recruitment of RNA polymerase and the second term is dependent on transcription factor activity: the probability that the \(i^{th}\) transcription factor is DNA-bound is \(p_{T, i}\), and the recruitment effect of the \(i^{th}\) transcription factor on the \(j^{th}\) gene is \(\Delta r_{i j}\). The \(\alpha\) and \(\Delta r\) values are computed prior to simulation based on gene expression values from conditions that modulate transcription factor activity.  Values for \(p_T\) are calculated as described in Table~\ref{table:transcription_pt}.


\begin{table}[!hbt]
\centering
\begin{tabular}{r l}
Transcription factor type & Promoter-bound probability \\
\hline
Zero-component system & \(p_T = 1\) if TF is present, \(0\) otherwise \\
One-component system & \(p_T = (T^*) / (T^* + T)\) \\
Two-component system & \(p_T = (T_p) / (T_p + T)\) \\
\end{tabular}
\caption[Formulas used to compute the probability that a transcription factor is promoter-bound.]{Formulas used to compute the probability that a transcription factor is promoter-bound. \(T^*\) is the active form of a one-component system transcription factor, while \(T_p\) is the phosphorylated form of a two-component system transcription factor, and \(T\) is the inactive or unphosphorylated form of a transcription factor.}
\label{table:transcription_pt}
\end{table}

\hspace{1cm}

\begin{algorithm}[H]
\caption{Equilibrium binding}
\label{equilibrium_binding_algorithm}
\SetKwInOut{Input}{Input}\SetKwInOut{Result}{Result}
    \Input{$c_m$ counts of molecules where $m = 1$ \KwTo $n_{molecules}$}
    \Input{$S$ matrix describing reaction stoichiometries where $S[i,j]$ describes the coefficient for the $i^{th}$ molecule in the $j^{th}$ reaction}
    \Input{$reactants_j$ set of indices for $c_m$ of reactant molecules that participate in the $j^{th}$ reaction}
    \Input{$product_j$ index for $c_m$ of the product molecule formed by the $j^{th}$ reaction}
    \Input{$f$ conversion factor to convert molecule counts to concentrations}
    \Input{$K_{d,j}$ dissociation constant where $j = 1$ \KwTo $n_{reactions}$}
    \textbf{1.} Dissociate all complexes in \(c\) into constituent molecules to get total reactants ($d$) since some reactants participate in multiple reactions: \\
    \Indp
    \SetInd{1.5em}{0em}
    $d = c$ \\
    \For{each ligand-binding reaction, j}{
        \For{each molecule, i}{
            $d_i = d_i + c_{product_j} \cdot S[i, j]$
        }
        $d_{product_j} = 0$
    }
    \Indm
    \textbf{2.} Find the number of reactions to perform ($n_j$) to minimize the distance from $K_{d,j}$, where $r$ is a positive integer and not greater than the total products that can be formed by the reactants: \\
    \Indp
    \For{each ligand-binding reaction, j}{
        $n_j = \mathop{\mathrm{argmin}}\limits_r \left| \frac{\prod\limits_{i \in \text{reactants}_j}(f\cdot (d_i - r))^{S[i,j]}}{f\cdot r} - K_{d,j} \right|$
    }
    \Indm
    \textbf{3.} Update counts ($c$) based on number of reactions that will occur. Starting from the dissociated counts, reactants will decrease by the number of reactions and their stoichiometry and one product will be formed for each reaction. \\
    \Indp
    $c = d$ \\
    \For{each ligand-binding reaction, j}{
        \For{each molecule in reactants$_j$, i}{
            $c_i = c_i - S[i, j] \cdot n_j$
        }
        $c_{product_j} = n_j$
    }
    \Indm
    \Result{Ligands are bound to or unbound from their binding partners in a fashion that maintains equilibrium.}
\end{algorithm}

\newpage
\begin{algorithm}[H]
\caption{Two-component systems}
\label{two_component_systems_algorithm}
\SetKwInOut{Input}{Input}
\SetKwInOut{Result}{Result}
\SetKwFunction{solveToNextTimeStep}{solveToNextTimeStep}

  \Input{$\Delta t$ length of current time step}
    \Input{$c_m$ counts of molecules where $m = 1$ \KwTo $n_{molecules}$}
    \Input{$k_A$ rate of phosphorylation of free histidine kinase}
    \Input{$k_B$ rate of phosphorylation of ligand-bound histidine kinase}
    \Input{$k_C$ rate of phosphotransfer from phosphorylated free histidine kinase to response regulator}
    \Input{$k_D$ rate of phosphotransfer from phosphorylated ligand-bound histidine kinase to response regulator}
    \Input{$k_E$ rate of dephosphorylation of phosphorylated response regulator}
    \Input{\solveToNextTimeStep{} function that solves two-component system ordinary differential equations to the next time step and returns the change in molecule counts ($\Delta c_m$)}

  \textbf{1.} Solve the ordinary differential equations describing phosphotransfer reactions to perform reactions to the next time step ($\Delta t$) using $c_m$, $k_A$, $k_B$, $k_C$, $k_D$ and $k_E$.\\
    \hspace{1cm} $\Delta c_m$ = \solveToNextTimeStep{$c_m$, $k_A$, $k_B$, $k_C$, $k_D$, $k_E$, $\Delta t$}

    \textbf{2. } Update molecule counts.\\
    \hspace{1cm} $c_m = c_m + \Delta c_m$

    \Result{Phosphate groups are transferred from histidine kinases to response regulators and back in response to counts of ligand stimulants.}
\end{algorithm}
\vspace{1cm}
\begin{algorithm}[H]
\caption{Transcription factor binding}
\label{tf_binding_algorithm}
\SetKwInOut{Input}{Input}\SetKwInOut{Result}{Result}
\SetKwFunction{randomChoice}{randomChoice}
  \Input{$c_a^i$ counts of active transcription factors where $i = 1$ \KwTo $n_{\text{transcription factors}}$}
    \Input{$c_i^i$ counts of inactive transcription factors where $i = 1$ \KwTo $n_{\text{transcription factors}}$}
    \Input{$P_i$ list of promoter sites for each transcription factor where $i = 1$ \KwTo $n_{\text{transcription factors}}$}
    \Input{$t_i$ type of transcription factor (either one of two-component, one-component, or zero-component) where $i = 1$ \KwTo $n_{\text{transcription factors}}$}
    \Input{\randomChoice{} function that randomly samples elements from an array without replacement}

  \For{each transcription factor, i}{
    \If{active transcription factors are present}{

    \textbf{1.} Compute probability \(p\) of binding the target promoter.

    \eIf{$t_i$ is zero-component transcription factor}{
    \-\hspace{1cm} transcription factor present $\rightarrow p_T = 1$\\
    \-\hspace{1cm} transcription factor not present $\rightarrow p_T = 0$
    }
    {
    \-\hspace{1cm} $p_T = \frac{c_a^i}{c_a^i + c_i^i}$
    }

    \textbf{2.} Distribute transcription factors to gene targets.\\
    \-\hspace{1cm} $P_i^{bound} =$ \randomChoice{from $P_i$ sample $p_T \cdot len(P_i)$ elements}

    \textbf{3.} Decrement counts of free transcription factors.
    }
    }

    \Result{Activated transcription factors are bound to their gene targets.}
\end{algorithm}


\begin{table}[H]
\hspace{16pt} \textbf{Associated data} \\\\
 \label{transcription_regulation_table}
 \begin{tabular}{p{2in} p{0.8in} p{0.9in} p{1in} c}
 \hline
 Parameter & Symbol & Units & Value & Reference \\
 \hline
Ligand::TF dissociation constant & $k_d = k_r/k_f$ & $\mu$M & [2e-15, 5e3] & See GitHub   \\
Free HK phosphorylation rate & $k_A$ & $\mu$M/s & [1e-4, 5e2] & See GitHub  \\
Ligand::HK phosphorylation rate & $k_B$ & $\mu$M/s & 1.7e5 & See GitHub  \\
Phosphotransfer rate from free HK-P to TF & $k_C$ & $\mu$M/s & 1e8 & See GitHub  \\
Phosphotransfer rate from ligand::HK-P to TF & $k_D$ & $\mu$M/s & 1e8 & See GitHub  \\
Dephosphorylation rate of TF-P & $k_E$ & $\mu$M/s & 1e-2 & See GitHub  \\
DNA::TF dissociation constant & $K_d$ & pM & [2e-4, 1.1e5] & See GitHub  \\
Promoter sites & $n$ & targets per chromosome & [1, 108] & See GitHub  \\
Fold-change gene expression & $FC$ & $log_{2}(a.u.)$ & [-10.48, 9.73] & See GitHub  \\

 \hline
\end{tabular}
\caption[Table of parameters for transcription regulation]{Table of parameters for equilibrium binding, two-component systems, and transcription factor binding Processes. HK: histidine kinase, TF: transcription factor, HK-P: phosphorylated histidine kinase, TF-P: phosphorylated transcription factor. Note in this and future tables we reference the source code for our model, which will be freely available at GitHub as noted in the main text.}
\end{table}


\begin{table}[H]
\hspace{16pt} \textbf{Associated files}
\begin{center}
 \scriptsize
 \begin{tabular}{c c c}
 \hline
 \texttt{wcEcoli} Path & File & Type \\
 \hline
\texttt{wcEcoli/models/ecoli/processes} & \texttt{equilibrium.py} & process \\
\texttt{wcEcoli/models/ecoli/processes} & \texttt{tf\_binding.py} & process \\
\texttt{wcEcoli/models/ecoli/processes} & \texttt{two\_component\_system.py} & process \\
\texttt{wcEcoli/reconstruction/ecoli/dataclasses/process} & \texttt{equilibrium.py} & data \\
\texttt{wcEcoli/reconstruction/ecoli/dataclasses/process} & \texttt{transcription\_regulation.py} & data \\
\texttt{wcEcoli/reconstruction/ecoli/dataclasses/process} & \texttt{two\_component\_system.py} & data \\
\texttt{wcEcoli/reconstruction/ecoli/flat} & \texttt{equilibriumReactions.tsv} & raw data \\
\texttt{wcEcoli/reconstruction/ecoli/flat} & \texttt{foldChanges.tsv} & raw data \\
\texttt{wcEcoli/reconstruction/ecoli/flat} & \texttt{tfIds.tsv} & raw data \\
\texttt{wcEcoli/reconstruction/ecoli/flat} & \texttt{tfOneComponentBound.tsv} & raw data \\
\texttt{wcEcoli/reconstruction/ecoli/flat} & \texttt{twoComponentSystems.tsv} & raw data \\
\texttt{wcEcoli/reconstruction/ecoli/flat} & \texttt{twoComponentSystemTemplates.tsv} & raw data \\
 \hline
\end{tabular}
\end{center}
\caption[Table of files for transcription regulation]{Table of files for transcription regulation.}
\end{table}

\newpage

\label{sec:references}
\bibliographystyle{plain}
\bibliography{references}

\end{document}
