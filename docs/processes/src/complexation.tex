\documentclass[12pt]{article}
\usepackage[ruled,vlined,noresetcount]{algorithm2e}

\topmargin 0.0cm
\oddsidemargin 0.2cm
\textwidth 16cm
\textheight 21cm
\footskip 1.0cm

\begin{document}

\baselineskip24pt

\paragraph{Complexation}
\label{sec:complexation}


\subparagraph{Model implementation.}
This process models the formation of all macromolecular complexes except for 70S ribosome formation, which is handled by \texttt{Translation}. Macromolecular complexation is done by identifying complexation reactions that are possible (which are reactions that have sufficient counts of all sub-components), performing one randomly chosen possible reaction, and re-identifying all possible complexation reactions. This process assumes that macromolecular complexes form spontaneously, and that complexation reactions are fast and complete within the time step of the simulation.\\

\begin{algorithm}[H]
\caption{Algorithm for macromolecular complexation}
\label{complexation_algorithm}
\SetKwInOut{Input}{Input}\SetKwInOut{Result}{Result}
\SetKwFunction{getPossibleReactions}{getPossibleReactions}
\SetKwFunction{chooseRandomReaction}{chooseRandomReaction}

  \Input{$c_{i}$ counts of molecules where $i = 1$ \KwTo $n_{molecules}$}
    \Input{$S$ matrix describing reaction stoichiometries where $S_{i,j}$ describes the coefficient for the $i^{th}$ molecule in the $j^{th}$ reaction}
    \Input{\getPossibleReactions function that takes $c_i$ and $S$ and returns all reactions that are possible}
    \Input{\chooseRandomReaction function that takes all possible reactions and returns one randomly chosen reaction}

  \While{possible reactions remaining}{
    
    \textbf{1.} Get all possible reactions ($r$)\\
    \-\hspace{1cm} $r = \getPossibleReactions (S, c_i)$
    
    \textbf{2.} Choose a random possible reaction ($r_{choice}$) to perform\\
    \-\hspace{1cm} $r_{choice} = \chooseRandomReaction(r)$
    
    \textbf{3.} Perform $r_{choice}$ by incrementing product counts and decrementing reactant counts
    }
    
    \Result{Macromolecule complexes are formed from their subunits.}
\end{algorithm}

\newpage
%\subparagraph*{Associated files} 
\textbf{Associated files}

\begin{table}[h!]
 \centering
 \scriptsize
 \begin{tabular}{c c c} 
 \hline
 \texttt{wcEcoli} Path & File & Type \\
 \hline
\texttt{wcEcoli/models/ecoli/processes} & \texttt{complexation.py} & process \\
\texttt{wcEcoli/reconstruction/ecoli/dataclasses/process} & \texttt{complexation.py} & data \\
 \hline
\end{tabular}
\caption[Table of files for complexation]{Table of files for complexation.}
\end{table}


\subparagraph{Associated data.}
Stoichiometric coefficients that define 1,023 reactions to form protein complexes from EcoCyc \cite{Keseler:2013di}.

\subparagraph{Difference from \emph{M. genitalium} model.}
This sub-model is implemented very similarly to the \textit{M. genitalium} model of complexation.  In the \textit{M. genitalium} simulations, however, the selection of a complexation reaction was weighted by a multinomial distribution parameterized by substrate availability rather than a uniform distribution.  We found that the choice of distribution had no major effect on behavior of the process. Additionally, the \textit{M. genitalium} simulations describe 201 macromolecular complexes, whereas over 5 times as many are implemented in the \textit{E. coli} model.

\newpage

\label{sec:references}
\bibliographystyle{plain}
\bibliography{references}

\end{document}